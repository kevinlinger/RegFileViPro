\section{Optimization}
\label{sec:opt}
\subsection{Knobs for optimization}
To simplify the optimization problem and manage complexity we restrict ourselves to a subset of design variables that can be tuned to generate an optimal SRAM virtual prototype that fits the user constraints. In this section, we describe our choice of knobs for optimization for each subcomponent.

\subsubsection{Bitcell}
For the basic case of a super-threshold, 6T SRAM, we use the bitcell provided in the design kit by the foundry. This bitcell is already carefully designed and optimized for area, power and performance and to simplify the problem we use this bitcell as is. For other kinds of SRAM, e.g. sub-vt, alternative bitcells etc., a bitcell is generated by the bitcell generator.

\subsubsection{CD}
The design variables that can influence the E and D of the CD are the device dimensions of the precharge, equalize, transmission gates, number of rows (NR) and V$_\text{DD}$. We assume the lengths of all devices to be the minimum/default length for the technology. We also make the width of the PMOS and NMOS in the transmission gate the same, as there is no need to balance the drives of each. We make the equalize transistor the same size as the precharge so that we have a uniform rectangle in the layout for the two PMOS and equalize transistor. Thus the variables are now reduced to the widths of the various transistors, NR, and V$_\text{DD}$.

By running the CD\_sizing test in device/TESTS, we find that the energy contribution of the precharge or the pass gate transistor does not change much for a given BL capacitance if it is upsized. This is because the load capacitance remains more or less constant as it is dominated by the BL capacitance. Thus energy is not a consideration when trying to size these devices. So we just upsize these devices till the delay lies on the knee of the delay-width pareto curve. For different technologies and for various values of BL capacitance, we find that a sizing of about 5-6 times the minimum width is optimal for these devices. This ensures the precharge is fast enough so it does not lie on the critical path during read while not being too area hungry. Also keeping these devices small helps reduce the load on the buffers in the timing block that drive the control signals to the gates of the precharge and transmission gates (e.g PCH, CSEL etc.).

Thus, in sum, the knobs for the CD are NR and V$_\text{DD}$, which are also global knobs that influence other component ED characteristics. All device lengths are the minimum values and all device widths are 6$\times$ the minimum. 

\subsubsection{SA}
The SA contribution to the total energy is not very significant. The SA design is based mainly on the failure probability. Using Mahmoodi's paper on sizing each device in the SA and how it affects the failure probability of the SA, we determine the sizes of the devices in the SA. 

First, we make the precharge and equalize transistors for the output nodes (OUT/OUTB), and the internal evaluation node (xin/xinb) minimum sized since these are small capacitances. The PFETs in the cross-coupled inverters are also minimum sized to reduce failure probability as is the NMOS footer. The NFETs in the cross-coupled inverters are made 3$\times$ the minimum width to reduce failure probability. This is the device that most affects the failure probability. The input NFETS that are driven by the RDWR/NRDWR also reduce the failure probability if they are upsized, but not as significantly as the cross-coupled NFETs do. So, we leave them minimum sized. Finally, the precharge PFETs for the RDWR/NRDWR are made 5$\times$ the minimum width as we did for the CD. Here we assume that the capitance being driven by these FETS is large compared to the self-load, since this node has the SA input, the CD transmission gate, the IO driver, and the long RDWR/NRDWR wire.

\subsubsection{Decoder}
\label{subsubsec:decoder}
The decoder E-D characteristics are influenced by the gate sizing, V$_\text{DD}$, and number of stages in the predecode and WL buffer chains. The knobs we use are the number of buffer chain stages and V$_\text{DD}$. We assume that the first gate of the buffer chain is minimum sized and the fan-out if 4, which is a good heuristic to reduce energy while paying only a small delay penalty (ref - Amrutur and Horowitz, Fast low-power decoders for SRAM). The optimal number of stage increases as the output WL load or the predecode wire load increases.

\subsubsection{Timing}
The timing block E-D are influenced by the buffer chains that drive the horizontal control and vertical predecode lines. The characteristics (e.g. sizing, number of stages) of these chains depend on the load that they are driving which depends on the number of rows/columns. So, this is one knob. The second knob is V$_\text{DD}$.
